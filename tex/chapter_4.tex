%# -*- coding: utf-8-unix -*-
%%==================================================
%% chapter_4.tex for SJTU Master Thesis
%%==================================================

\chapter{基于智能控制策略的时钟伺服系统}
在第\ref{chap:statistical_delay}章中,本人从链路传输延时的角度,通过深入分析实际工业环境下对链路传输延时的影响,通过对链路传输延时建立细致的数学模型,并从数学统计的角度对其中各个影响因素提出了有针对性的解决方案。

至此,我们已经能够利用上述算法综合获得稳定和精确的链路延时。但是,仅仅获取稳定精确的链路延时并不是时钟同步的终点。而且,在现实IEEE1588协议运行中,之所以同步精度难以达到亚微秒的同步精度,有很重要的因素是从时钟校正策略的选取。所谓时钟同步,最核心一环自然是从时钟根据计算数据对本地进行校正来实现同步,即时钟伺服系统。

\section{时钟伺服系统研究}
对从时钟而言,如果想要实现与主时钟同步,就必须依靠前面提到的同步算法及所计算出来的链路时延值,从而对从时钟自身的相位和频率进行校正,以有效消除与主时钟之间的偏差。不过,由于当前的很多工业平台并非实时操作系统,所以一般而言都需要利用中断机制来触发从时钟进行校正。但是,在该中断过程中造成的从时钟漂移抖动会长期破坏同步精度,严重影响同步过程的实时性。

因此,为了提高从时钟校正过程的实时性和快速性,我们需要设计一套比较良好的时钟伺服系统来对从时钟进行校正。

\subsection{通用PTP时钟伺服系统介绍}
---插图,时钟伺服系统

一种常见的PTP时钟伺服系统如图所示。从左到右表明了数据从PTP协议引擎逐渐流向从时钟。在该系统中,PTP协议引擎通过周期性获取主从偏差和从主偏差,将这两个偏差值通过均值滤波及低通滤波器获取到当前offset值。然后,将当前offset值与采样周期数据共同传递给PI控制器,由PI控制器来对从时钟进行校正以实现主从同步。该PI控制器由比例(P)和积分(I)两个环节共同构成,其中,比例环节可以用来消除主从时钟间的相位偏差,而积分项则可以用来消除系统的稳态误差,即消除主从时钟间的频率差。

上述这样一套较为完整的时钟伺服系统可以较好的实现时钟同步,而且,由于传统的PI控制算法相对而言比较简单、鲁棒性良好且可靠性较高,所以传统的PI控制器在工业控制领域里应用十分广泛。

但是,传统的PI控制器也有其固有的弊病,那就是其比例积分参数的调整往往需要依靠经验来设定。对于处于复杂网络环境中的时钟同步系统而言,仅仅依靠经验来整定控制参数的PI控制器几乎无法应对任何形式的复杂多变的环境,尤其是同步系统中的网络环境中存在多种非线性、时变性等不确定性因素时,被控对象特性常常会随着时间发生变化。例如工业中充当从时钟的PTP设备,很容易会随着时间慢慢精度降低,并且长期受到工业干扰导致设备性能逐渐下降。这些从时钟的变化都会导致传统的PI控制器在时钟伺服系统中无法达到良好的控制效果。

所以,为了使得时钟伺服系统能够适应工业网络环境变化,尤其是应对时变性非常强的网络传输环境,本人在下文从智能自适应控制的角度出发,结合同步系统特性进行深入分析,并且提出基于神经网络的PI控制器策略作为伺服系统中的控制环节,从而来弥补传统PI控制器无法适应复杂网络环境的缺陷。

\section{基于神经网络的PI控制策略}
根据上节可以知道,由于传统PI控制器无法适应网络传输环境和被控对象的时变性,我们需要采用一种能够自动根据当前网络环境而调整的控制策略。而通过调查知道,当前比较流行的智能控制方法是一种很好的选择。




\subsubsection{}