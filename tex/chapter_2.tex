%# -*- coding: utf-8-unix -*-
%%==================================================
%% chapter_1.tex for SJTU Master Thesis
%%==================================================

\chapter{IEEE1588协议原理及相关研究}
\label{chap:1588_theory}
本章主要介绍1588的同步原理,如最佳主时钟算法、同步算法。同时,介绍1588在实际使用中出现的问题和当前的研究方案。并提出这些研究方案中存在的缺陷。

\section{IEEE1588协议原理}

\subsection{最佳主时钟算法}
\label{sec:1588_theory_bmc}
最佳主时钟算法

\subsection{时钟同步算法}
\label{sec:1588_theory_sync}
时钟同步算法

\section{IEEE1588实际应用中存在的问题}
\subsection{链路延时不对称}
\label{sec:1588_problem_1}
在1588协议中,当计算主从时钟偏差offset时,需要先获取链路传输延时delay。在协议中该延时的计算方法是直接假设前后两次传输延时相等,从而计算出offset。然而实际上,前后两次的传输路径往往并不对称,由于传输漂移抖动、报文排队堵塞、拓扑结构变动都会导致往返的两次延时不一致,所以说,真实的传输延时也不能简单的假设相等。

\subsection{时间戳的精确度}
\label{sec:1588_problem_2}
由于在协议栈中传递所带来的延时,导致物理层之上的时间戳往往不能真实反映报文的发送时间或接收时间,若直接使用这种时间戳来进行计算,那么从根本上就带来了误差。

\subsection{从时钟稳定性}
\label{sec:1588_problem_3}
根据1588协议标准,对从时钟而言,一旦计算出offset值边直接进行校正,这样往往会导致从时钟无法稳定。




