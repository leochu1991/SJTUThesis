%# -*- coding: utf-8-unix -*-
%%==================================================
%% chapter_6.tex for SJTU Master Thesis
%%==================================================

\chapter{总结与展望}
行文至此,本人已经细致地对时钟同步技术及IEEE1588协议的相关问题进行了深入的研究和探讨,同时本人也创新性地从数学统计角度入手,提出了“动态阈值法”、“基于固定时间窗的实时动态监控算法”等,用来处理报文链路传输时延变化。同时也对于主从时钟源晶振漂移进行了分析,并从统计角度给出了相关的解决方案,而且通过文末的仿真能发现这些算法不仅能够在一定程度上提高系统的同步精度,而且能提高同步系统的稳定性和鲁棒性,对于工业环境下复杂的网络环境有良好的动态适应性。

下面,对于本文进行总结和展望。

\section{总结}
首先,由于当前分布式系统等应用越来越广泛,而其中时钟同步问题也已经成为了束缚分布式系统发展和实现很多实时性需求的一大瓶颈。而为了达到工业系统中所需要的时钟同步精度,我们通过分析了解发现遗忘的时钟同步方法如GPS、NTP、SNTP等技术均由于各种原因,如国防安全性、时钟同步精度和稳定性等,而无法满足当前工业环境对时钟同步的需求。而IEEE1588精密时钟同步协议由于其自身的易部署性、价格适中、同步精度高的多种特性使得其成为了当今非常重要且应用广泛的时钟同步技术。

然后,通过深入了解发现,IEEE1588时钟同步协议在实际运行中,并没有达到理想的微秒级甚至亚微秒级的时钟同步精度。于是,本人通过深入分析协议内容和实际运行背景,提炼出其中对同步精度破坏最大的两个因素:链路传输延时的随机性和变化性;从时钟校正策略。针对前者,本人创新性地从数学统计角度出发,把链路传输延时数据作为样本,并且根据其数学特性分别提出了“最小二乘线性回归算法”、“动态阈值法”和“基于固定时间窗的实时监控算法”等多种算法,分别对链路传输延时中的固有随机抖动、“暂时性”时延突变和“持久性”时延变化进行优化;然后,我们通过对主从时钟源晶振漂移进行了分析,提出了统计方法来进行补偿,另外还针对从时钟校正策略,从神经网络及PID控制角度出发,探索了这种控制策略的可行性。

在文章最后,我们利用stateflow工具,在matlab平台上搭建了一套完整的时钟同步仿真系统。本人利用该系统来对上文所提算法进行验证,通过最后的结果可以看出所提算法不仅能够改善同步系统的时钟同步精度,而且还能够充分应对链路中发生的由于拓扑结构变化等因素导致的“持久性”时延变化,这是当前很多同步算法并不具备的。除此之外,对于第四章对神经网络PID控制策略的可行性探究,我们也从仿真结果看出这种策略确实存在一定的应用价值和应用前景。

相信本人在上面所作的工作和研究成果,能够为之后的研究者提供一条不一样的思路,而且这些仿真结果能够为他们提供一定的实验依据。 

\section{展望}
在上文中,本人通过对IEEE1588时钟同步协议进行深入研究,发现了其中存在的严重问题,并且也提出了相应的解决方案,通过最后的仿真结果可以看出,这些方案确实能够改善同步系统的同步精度、稳定性和鲁棒性。

但是,这并不意味着这些方案已经没有缺陷了。

本人认为,后续的研究者可以从以下两个方面着手,进一步对本人所提的算法进行优化和改进以继续提高工业同步系统的同步精度和稳定性等指标。
\begin{enumerate}[noitemsep,topsep=0pt,parsep=0pt,partopsep=0pt]
	\item 由于本文针对链路延时是从数学统计角度出发进行研究,所以,这意味着从时钟必须能够存储一定数量的样本数据。当然,虽然这些数据都非常小,每个样本仅仅只是存储一个时延样本值,不过,对于存储空间有限的工业交换机设备而言,这些空间仍然是宝贵的,所以,后面可以探讨如何更有效的存储样本数据,如何及时把过时样本数据清空以保证硬件层能够有能力存储这些样本值;
	\item 对于时钟伺服系统,本文仅仅是从神经网络PID控制的角度进行了可行性验证,然后,具体的实现方法还有待改进。而且,神经网络的优势是能够对环境有很好的自适应性,这能够提供时钟同步系统对工业复杂网络环境的适应。但是,神经网络的缺陷是需要训练,而训练势必会导致较长的耗时,所以,如何衡量耗时和性能是一个重要的研究课题。
\end{enumerate}

最后,随着分布式系统和实时性需求的不断发展,时钟同步将会越来越影响到所有系统性能的关键一环,而当前实现时钟同步最有效的便是IEEE1588协议,因此对该协议的研究具有很广阔和长远的研究价值。而对协议而言,越来越高的时钟同步精度和系统稳定性将是持久的研究课题。