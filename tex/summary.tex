%# -*- coding: utf-8-unix -*-
%%==================================================
%% conclusion.tex for SJTUThesis
%% Encoding: UTF-8
%%==================================================

\begin{summary}
研究生期间在导师带领下,开始逐步接触时钟同步技术和IEEE1588精密时钟同步协议。在了解中发现,目前工业领域越来越多的应用了IEEE1588精密时钟同步协议来提高分布式系统的同步性能,所以也在研究学习中对时钟同步技术和IEEE1588协议本身产生兴趣。然后通过不断的阅读各种文献和深入阅读IEEE1588协议内容,对时钟同步技术有了全面的了解。

不过,慢慢就发现了IEEE1588精密时钟同步协议在实际应用中仍然存在的很多缺陷和问题。正是这些问题的存在导致了实际的运行效果不尽如人意,其同步精度也往往很难达到理想的目标。因此,决定着手深入研究这些问题的根源及其解决方案。

在导师的指引下,文章中创新性地从数学统计角度来深入分析这些问题,并且提出了一套自己的解决方案,这一套统计算法覆盖了报文固有抖动、排队堵塞等导致的暂时性时延突变、拓扑结构变化导致的持久性时延变化和从时钟频率漂移等多种破坏同步精度的因素。最后,通过搭建的stateflow仿真平台上对所提算法进行验证,可以发现这些算法不仅能有效提高系统的同步精度,还能改善同步系统应对复杂工业环境的稳定性和自适应性。

然后,通过这篇论文的撰写,又重新梳理了之前的思路,并试图以尽可能准确的语言和描述把之前的研究工作及成果展示给大家。同时,也将研究过程进行记录,以方面后续的研究人员在此基础上能够继续探究,不断改良时钟同步系统的精度,使得IEEE1588协议能够发挥其理想的效果,具备更大的实际应用价值。

最后,对时钟同步技术的研究培养了自身的科研习惯,能够从更加理性的角度去思考问题,也学会了通过阅读国内外文献来扩展自己的视野和优化自己的思维方式。希望以后能够继续保持这种谨慎认真的态度来面对未来的工作和学习。

\end{summary}
