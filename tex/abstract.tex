%# -*- coding: utf-8-unix -*-
%%==================================================
%% abstract.tex for SJTU Master Thesis
%%==================================================

\begin{abstract}
时钟同步问题来源于工业中分布式系统的发展和实时性任务的需求,核心问题是通过同步方法来实现各个子系统之间的时钟一致性,传统的GPS同步方法和NTP(网络时钟同步协议)等协议由于价格昂贵、同步精度差等原因满足不了工业需求,IEEE1588精确时钟同步协议协议是目前时钟同步领域最好的同步方法,不过其在实际使用中由于链路延时等因素影响难以达到纳秒甚至亚纳秒的同步精度以及无法保证从时钟的稳定性。

本文从统计分析的角度研究了链路延时、时钟伺服系统及时钟频率漂移补偿的相互关系,以提高IEEE1588时钟同步系统精度及稳定性,主要研究内容如下:

(1) 基于链路延时的统计特性,研究了动态阈值法和滑动时间窗的实时监控算法,以提高同步精度;

(2) 在时钟伺服系统中,针对链路时延的阶跃突变干扰采用多模型PID控制策略,以优化从时钟的稳定性,减少从时钟的频繁抖动;

(3) 对于时钟频率漂移问题,通过监测PTP报文样本是否发生链路堵塞等来进行报文过滤,只选取正常链路传输的PTP样本来进行校正,从而提高频率偏差计算的准确度;

% (4) 通过对比边界时钟与透明时钟差异性来对统计算法进行调整,以使本文所提算法能够应用到透明时钟场合中;

(4) 通过stateflow时钟同步仿真系统进行仿真分析。仿真结果表明不仅时钟同步系统对多种链路延时抖动的适应性得到了提高,而且整个系统的同步精度和稳定性也得到了提高。

\keywords{\large 时钟同步 \quad 统计算法 \quad 频率补偿 \quad 时钟伺服系统 \quad  \quad stateflow}
\end{abstract}

\begin{englishabstract}

Time synchronization problem originates from the requirement of developing distributed system and real-time tasks. The core problem is to achieve the time consistency among all the subsystems through time-synchronization method. The traditional GPS method and NTP protocol are not able to meet current industrial demand due to their high-cost and unpredictable message latency. Currently, IEEE1588 protocol is the best choice for time-synchronization problem, However, it often cannot achieve the precision level of sub-microsecond or nanosecond and it's not able to guarantee the stability of slave clock in the industrial reality.

From the perspective of mathematical statistics, the objective of this paper is to promote a systematically statistical method to improve the time-synchronization precision and stability. The contributions are shown in the following.

(1) Through the mathematical analysis for transmitting message latency, I decomposed latency into inherent jitter, temporary latency mutation and persistent latency change. In order to improve accuracy, dynamic threshold method and fixed-window real-time detection method are promoted.

(2) In clock servo system, multi-model PID controller strategy is promoted for improving the slave-clock stability when step-mutation occurs. So slave clock system will become more smooth due to the less jitter.

(3) By filtering PTP messages based on the detection of link blocking, qualified PTP messages will be choosed for calcuation of master-to-slave frequency offset.

% (4) By comparing boundary-clock with transparent-clock, the statistics methods are promoted in order to make our statistical methods able to be applied into the occasion of transparent-clock device.

(4) Based on stateflow time-synchronization simulation system, all the promoted methods above get validated and verified. The simulation results show us those methods above can improve the adaptive ability of time-synchronization system for multiple types of latency changes, and the synchronization precision and stability of whole system as well.

\englishkeywords{\large clock synchronization,  statistical method, frenquency compensation, clock servo, stateflow}
\end{englishabstract}

