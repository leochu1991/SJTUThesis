%# -*- coding: utf-8-unix -*-
%%==================================================
%% abstract.tex for SJTU Master Thesis
%%==================================================

\begin{abstract}
时钟同步问题来源于工业中分布式系统的发展和实时性任务的需求,核心问题是通过同步方法来实现各个子系统之间的时钟一致性,传统的GPS同步方法和NTP\footnote{网络时钟同步协议}等协议由于价格昂贵、同步精度差等原因满足不了工业需求,IEEE1588\footnote{精密时钟同步协议}协议是目前时钟同步领域最好的同步方法,不过其在实际使用中由于链路延时等因素影响难以达到亚微秒甚至纳秒同步精度,且从时钟稳定性没有保证。

本文从统计角度出发,对链路延时、时钟伺服系统及时钟频率漂移补偿进行深入研究,旨在提供一种系统性的统计方法来提高IEEE1588时钟同步系统精度及稳定性,主要工作如下:

(1) 通过对链路延时成分进行数学特性分析,拆解成固有抖动、暂时性时延突变和持久性时延变化,从统计角度提出动态阈值法与基于固定时间窗的实时监控算法等以提高同步精度;

(2) 在时钟伺服系统中,针对链路时延的阶跃突变干扰,提出多模型PID控制策略,以优化从时钟的稳定性,减少从时钟的频繁抖动以保证整个系统的平稳运行;

(3) 对于时钟频率漂移问题,通过监测Sync报文样本是否发生链路堵塞等来进行报文过滤,只选取正常链路传输的Sync样本来进行校正,从而提高频率偏差计算的准确度;

(4) 通过对比边界时钟与透明时钟差异性来对统计算法进行调整,以使本文所提算法能够应用到透明时钟场合中;

(5) 通过stateflow时钟同步仿真系统,对于上文所提算法进行仿真分析。仿真结果表明上述算法不仅能够提高时钟同步系统对多种链路延时抖动的适应性,还能提高整个系统的同步精度和稳定性。

\keywords{\large 时钟同步 \quad 统计算法 \quad 频率补偿 \quad 时钟伺服系统 \quad  \quad stateflow}
\end{abstract}

\begin{englishabstract}

Time synchronization problem originates from the requirement of development of distributed system and real-time tasks. The core problem is to achieve the time consistency among all the subsystems through tims-synchronization problem. The traditional GPS method and NTP protocol aren't able to meet current industrial demand in cause of their high-cost and unpredictable message latency. Currently, IEEE1588 protocol is the best choice for time-synchronization problem, However, it cann't achieve sub-microsecond level or nanosecond level precision and it's not able to guarantee the stability of slave clock.

From the perspective of mathematical statistics, the aim of this paper is to promote a systemic statistical method to improvement the time-synchronization precision and stability. The contribution mainly include

(1) Through the mathematical analysis for transmitting message latency, Decomposed latency into inherent jitter, temporary latency mutation and persistent latency change. Then dynamic threshold method、fixed-window real-time detection method for accuracy improvement are promoted.

(2) In clock servo system, Multi-model PID controller strategy is promoted for improving the slave-clock stability when step-mutation occurs. So less jitter of slave clock system will become more smooth.

(3) By filtering Sync message based on the detection of link blocking, qualified Sync messages will be choosed for calcuation of mast-to-slave frequency offset.

(4) By comparing bound-clock with transparent-clock, the statistics methods, which are promoted in this paper, will also be able to applied into the occasion with transparent-clock device.

(5) Based on stateflow time-synchronization simulation system, all the promoted methods above get validation and verfication. And the simulation result show us those methods above can not only improve the adaptive ability of time-synchronization system for multi latency change, but also the synchronization precision and stability of whole system.

\englishkeywords{\large Clock Synchronization,  statistical method, frenquency compensation, clock servo, stateflow}
\end{englishabstract}

