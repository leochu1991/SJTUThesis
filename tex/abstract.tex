%# -*- coding: utf-8-unix -*-
%%==================================================
%% abstract.tex for SJTU Master Thesis
%%==================================================

\begin{abstract}
在工业领域中,随着计算机通信技术的不断发展,越来越多的分布式系统被应用在了工业现场。而这些分布式系统由于物理范围的不断扩张和越来越多的实时性任务如数据采集、运行控制的需求,使得整个系统对于各个分布式节点的时钟同步精度提出了愈来愈高的要求。

所以,为了应对这些时钟同步需求,工业领域依次出现了不少时钟同步方法。例如GPS同步方法和NTP\footnote{网络时钟同步协议}、SNTP\footnote{简单网络时钟同步协议}等同步协议的出现。不过,由于工业现场中复杂的网络状况、工业时钟设备随环境变化和干扰而稳定性不高,而且网络报文传输中出现的随机时延变化等多种因素的存在,导致了实际同步精度并不理想,只能达到普通的毫秒级别。而GPS同步方法由于价格高昂和国防安全性的缺陷导致其无法长时间大面积应用于工业领域。

因此,在工业系统对时钟同步精度提出的要求越来越高,甚至达到微秒级别甚至亚微秒级别的背景下,IEEE1588精密时钟同步协议应运而生。该协议于2002年由网络精密时钟同步委员会在IEEE标准委员会的支持下提出,并于2008年发布了第二版。这套协议和过去的协议相比,从易部署性、国防安全性和时钟同步精度等多方面有了改进,而且根据协议内容,时钟同步精度能够达到微秒级别甚至更高,因此,本文选IEEE1588协议为核心研究对象。

然而,在实际的工业运行中我们发现,IEEE1588同步协议仍然存在严重的问题。这些问题导致时钟同步精度难以达到理想的微秒级甚至亚微秒级的同步精度,这不仅严重束缚了该协议在工业系统中的应用,也使得更多分布式系统对时钟同步的需求更加迫切。因此,为了充分发挥IEEE1588在工业实际系统中的价值,本文对该协议进行了深入分析,结合工业实际来探究其中破坏时钟同步精度的各种因素,然后在第三章钟创新性地从数学统计角度出发对传输时延进行深入研究,提出“动态阈值法”、“基于固定时间窗的实时监控算法”等统计方法进行优化。以时延抖动为例,由于该因素的随机性和不可预知性,一直以来都很难有很好的应对方案,然后本文创新性地从统计角度出发,把时延数据作为样本利用相关的数学方法来进行处理,从而明显提高了时钟同步的精度和稳定度。另外,本文第四章还从基于神经网络的PID从时钟控制策略角度进行了一定的探索,以期提高同步系统的鲁棒性和稳定性。

文末,我们自己搭建了一套基于stateflow的时钟同步系统,并且在该仿真系统上进行了算法验证,通过仿真结果可以看出本文所提算法使得时钟同步精度得到明显的改善。另外,相信本文创新性地从统计角度切入,所得到研究结果会为后面更多学者提供进一步深入探索的思路和依据。

\keywords{\large 时钟同步 \quad IEEE1588 \quad 统计方法 \quad 时钟伺服系统 \quad stateflow}
\end{abstract}

\begin{englishabstract}

Abstract 

\englishkeywords{\large Time Synchronization, IEEE1588, statistical method, clock servo system, stateflow}
\end{englishabstract}

