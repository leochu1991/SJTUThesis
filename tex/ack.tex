%# -*- coding: utf-8-unix -*-
\begin{thanks}

在交大研究生的这两年时间里,我从一名从没接触过科研的普通学生起步,开始接触交大优秀的教师,并在研一期间学习了非常多自己感兴趣的知识,而每一堂课的老师都能够耐心回答我所提出的疑惑,使得本人的知识面有了非常大的拓宽。所以我很感谢交大有一批优秀的教师为我们传道解惑。

进入实验室后,开始跟随导师完成实验室的项目,依次接触过软件验证、核电电路图绘制和时钟同步技术研究等多个工作。在这些工作中,导师通过不断指导给予了我很多帮助,也让我从一名初学者逐渐学习提高。尤其是在时钟同步技术方面的深入研究历程极大提高了我的科研能力,让我感受到了作为一名科研工作者应有的认真和谨慎。所以非常感谢导师对我的耐心传授和方向指引,使得我真正感受到作为一名研究生的价值。

然后,还要感谢父母在读研期间对自己的关心和鼓励,尤其是当自己生病时陪伴左右。这些都是本人不断前进的动力来源。另外,实验室的师兄弟们也给予了我很多的关心和帮助,我也要感谢他们对本人的照顾和帮助。

最后,感谢交大这所学校,本人在这里感受到了一所综合学府的学术、上进和踏实,也在这里完成了本科生到研究生的蜕变。

无论是从心智还是情感,这里的收获都将让我受益匪浅。

感谢!
\end{thanks}
